\documentclass[12pt]{article}

\usepackage{xcolor}
\usepackage[UKenglish]{isodate}% http://ctan.org/pkg/isodate
\usepackage[paperwidth=8.5in,paperheight=11in,margin=0.75in]{geometry} 
\usepackage[USenglish]{babel}
\usepackage{hyperref}
\usepackage[activate={true,nocompatibility},final,tracking=true,kerning=true,spacing=true,factor=1100,stretch=10,shrink=10]{microtype}

\usepackage{multirow}
\usepackage[T1]{fontenc} 
\usepackage{fourier}

\usepackage{enumerate,isomath,hyperref}
\usepackage{upgreek,comment}
\usepackage{graphicx}
\usepackage[super]{nth}
\usepackage{amsmath}

\newenvironment{alphalist}{
  \begin{enumerate}[(a)]
    \addtolength{\itemsep}{-0.5\itemsep}}
  {\end{enumerate}}
  \cleanlookdateon% Remove ordinal day reference
  \newcommand{\RomanNumeralCaps}[1]
      {\MakeUppercase{\romannumeral #1}}


      \usepackage{amstext} % for \text macro
      \usepackage{array}   % for \newcolumntype macro
      \newcolumntype{L}{>{$}l<{$}} % math-mode version of "l" column type
      
      \newcommand{\dom}{\mathrm{dom}} 
      \newcommand{\range}{\mathrm{range}} 
      \newcommand{\zero}{\mathrm{zero}} 
      \newcommand{\reals}{\mathbf{R}} 
      \newcommand{\integers}{\mathbf{Z}} 
      \newcommand{\ssep}{\mid}
      \newcommand{\arcsec}{\mathrm{arcsec}}
      \newcommand{\arccsc}{\mathrm{arccsc}}
      \newcommand{\arccot}{\mathrm{arccot}}
\begin{document}
\section*{Functions}

We'll use some standard mathematical notation that allows us to express facts compactly and precisely.  Much of this notation isn't in our book, but
we'll use it because it's not hard to remember and it is convenient. For additional standard notation, see \url{https://en.wikipedia.org/wiki/ISO_31-11}.

\vspace{0.1in}

\begin{tabular}{|l | l |} \hline 
\textbf{Notation}& \textbf{Meaning} \\ \hline
    $\dom(F)$  &  The \emph{domain} of a function $F$;   thus $\dom(F)$  is the set of all \emph{inputs} to \\
                         & the function $F$. \\  \hline
    
    $\range(F)$  & The \emph{range} of a function   $F$ is the set $\range(F)$; thus    thus $\range(F) $ is  \\
                          & the set of all \emph{outputs} to the function $F$.\\ \hline
                          
   $F : A \to B$ & This means that $F$ is a function whose domain is   the set $A$ and whose range is a \\
                           & \emph{subset} of the set $B$. \\ \hline        
                           
   $F(S)$     & When $F$ is a function and $S$ is a subset of its domain, we define \\
                    &   \( F(S) = \{F(x) |  x \in \dom(F) \}  \).\\ \hline                                    
                          
 $ x \in A \mapsto F(x)$  & Defines a function whose domain is the set $A$ and whose formula is $ F(x)$. \\  \hline
 
$  C_A $ &  The set of functions that are continuous on the set $A$. \\  \hline

$ C_A^1$  &  The set of functions that are continuous and whose first derivatives \\ 
                   &       are continuous on the set $A$. \\  \hline

$ C_A^n$  &  The set of functions that are continuous and whose first   through  \\
                   & $n^{\mbox{th}}$  derivatives are continuous on the set $A$. \\  \hline        
                   

$ C_A^\infty$  &  The set of functions that are continuous and whose derivative of all  \\
                   & orders are continuous on the set $A$. \\  \hline                                
                   
\end{tabular}

\subsubsection*{Examples}

\begin{alphalist}
\item Every real number is a valid input to the natural exponential function \(\exp\); thus \(\dom(\exp) = \reals\).

\item The set of outputs to the  natural exponential function is \((0,\infty) \); thus \(\range(\exp) = (0,\infty)\).

\item The domain of the sine function is $\reals$ and every output of sine is in $\reals$; thus $\sin : \reals \to \reals$.


\item The domain of the sine function is $\reals$ and every output of sine is in $[-1,1]$; thus $\sin : \reals \to [-1,1]$.
It's somewhat confusing that both $\sin : \reals \to \reals$ and $\sin : \reals \to [-1,1]$, remember that the notation
$F : A \to B$ means that $B \subset \range(F)$ and \emph{not} $B =\range{F}$.


\item  We have $\sqrt{[0,4] } = [0,2]$. And $\sin([0,\uppi]) = [0,1]$


\item The domain of the natural logarithm $\ln$ is the interval $(0,\infty)$ and every output of \(\ln\) is a real number; thus
$\ln : (0,\infty) \to \reals$.


\item $x \in [-1,1] \mapsto x^2$ defines a function whose domain is the set $[-1,1]$ and whose output is the square of
the input.


\item Since the sine function is continuous on \(\reals\), we have $\sin \in C_\reals$.

\item Since the sine function is infinityly differentiable on  \(\reals\), we have $\sin \in C_\reals^\infty$.

\item Since the sine function is continuous on \([0,2 \uppi]\), we have \(\sin \in C_{[0,2 \uppi]}\).

\item The square root function \(\sqrt{\phantom{x}}\) is continuous on \([0,\infty)\), but its derivative is not; thus we have
\(\sqrt{\phantom{x}} \in C_{[0,\infty)} \) and  \(\sqrt{\phantom{x}} \notin C_{[0,\infty)}^1 \) .
\end{alphalist}

\subsubsection*{Theorems}

\begin{alphalist}  

\item $C_A^2  \subset C_A^1 .$

In words this says that if a function has a continuous second derivative on an interval $[a,b]$, it has a continuous first derivative on the interval $[a,b]$.
Similarly, we have $\cdots C_A^4 \subset C_A^3  \subset C_A^2  \subset C_A^1 $.

\item $F \in C_{[a,b]}\land G \in C_{[a,b]} \implies F+G \in C_{[a,b]}$. 

In words, this says that the sum of  functions that are continuous on an
interval $[a,b]$ is is continuous on the interval $[a,b]$.

\item $F \in C_{[a,b]}\land G \in C_{[a,b]} \implies F G \in C_{[a,b]}$. 

In words, this says that the product of  functions that are continuous on an
interval $[a,b]$ is is continuous on the interval $[a,b]$.

\item $F \in C_{[a,b]}\land G \in C_{[a,b]} \land 0 \notin G([a,b])  \implies F / G \in C_{[a,b]}$. 

In words, this says that if functions $F$ and $G$ are continuous on an interval $[a,b]$ and $G(x) \neq 0$ for all \(x \in [a,b]\), then $F/G$ is continuous on $[a,b]$.



\end{alphalist}

\end{document}