\documentclass[12pt,fleqn]{article}
\usepackage{array}
\usepackage{xcolor}
\usepackage{fleqn}
\usepackage[USenglish]{isodate}% http://ctan.org/pkg/isodate
\usepackage[letterpaper,paperwidth=8.5in,paperheight=11in,margin=0.75in]{geometry} 
\usepackage[USenglish]{babel}
\usepackage{hyperref}
\usepackage[activate={true,nocompatibility},final,tracking=true,kerning=true,spacing=true,factor=1100,stretch=10,shrink=10]{microtype}
\usepackage{tcolorbox}

\usepackage{multirow}
\usepackage[T1]{fontenc} 
\usepackage{fourier}

\usepackage{enumerate,isomath,hyperref}
\usepackage{upgreek,comment}

\usepackage{graphicx}
%\usepackage[super]{nth}
\usepackage{amsmath}

\newenvironment{alphalist}{
  \begin{enumerate}[(a)]
    \addtolength{\itemsep}{-0.5\itemsep}}
  {\end{enumerate}}
  \cleanlookdateon% Remove ordinal day reference
  \newcommand{\RomanNumeralCaps}[1]
      {\MakeUppercase{\romannumeral #1}}


      \usepackage{amstext} % for \text macro
      \usepackage{array}   % for \newcolumntype macro
      \newcolumntype{L}{>{$}l<{$}} % math-mode version of "l" column type
      
      \newcommand{\dom}{\mathrm{dom}} 
      \newcommand{\range}{\mathrm{range}} 
      \newcommand{\zero}{\mathrm{zero}} 
      \newcommand{\reals}{\mathbf{R}} 
      \newcommand{\integers}{\mathbf{Z}} 
       \newcommand{\rationals}{\mathbf{Q}} 
      \newcommand{\ssep}{\mid}
      \newcommand{\arcsec}{\mathrm{arcsec}}
      \newcommand{\arccsc}{\mathrm{arccsc}}
      \newcommand{\arccot}{\mathrm{arccot}}
      


\newcommand\showdiv[1]{\overline{\smash{)}#1}}
      
      \title{How do you want me to simplify this?}
\begin{document}

\maketitle
\begin{quote}
\emph{My (admittedly perverse) answer is that ``to simplify''  means to write an equivalent expression that the instructor/marker likely wants or expects as an answer. It is an exercise in mind-reading.} \\  \phantom{xxxxx} \hfill   {\mbox{\sc B.\ S.\ Thomson}\footnote{\tiny Professor Emeritus at Simon Fraser University;  see  \url{https://www.quora.com/What-does-it-mean-to-simplify-an-expression?share=1}} }
\end{quote}


Undoubtedly, at least one math teacher has told you that you \emph{must} simply 
your answers. And maybe you have been frustrated by not earning full
credit for an answer that was correct, but not expressed in \emph{exactly}
the form required by the teacher. If you asked your teacher \emph{exactly} what it 
means to be simplified that means, it's unlikely that you got a clear 
answer. 

We'll attempt to give a guide to what it means to be simplified, but we'll
also explain some of goals of simplification and in doing so uncover
the reasons why your teacher may have not given you a particularly good
answer to the reasonable question ``How do you want me to simplify this?''

\subsection*{Goals of simplification}

Algebraic equality is such an important concept in mathematics, it would be nice to be able to easily decide if any two expressions are algebraically
the  same. Ideally, every pair of expressions that are algebraically the same would simplify to exactly the same expression. That way, visually we could
tell if two expressions are equal. With such a scheme, to determine if two expressions are algebraically the same, we need only to simplify them 
and visually compare them. If their simplified forms are identical, the unsimplified expressions are algebraically the same. From the viewpoint
of a paper grader, such a simplification scheme is ideal--visually it's possible to decide if an answer is correct or not.

Expressions that are algebraically the same are also known as \emph{semantically identical}, and expressions that are visually the same are \emph{syntactically identical}.  If a simplification scheme converts all semantically identical expressions into syntatically identical expressions,
we say the simplified form of an expression is called its \emph{canonical form}.  In this context, the word ``canonical'' is fancy word that means ``standard.''

Assuming that the canonical representation of zero is itself, it follows that every expression that is algebraically equivalent to zero will
simplify to zero. And knowning that an expression is nonvanishig keeps us from making errors such as
\begin{equation}
  \left(\frac{1}{\sqrt{2}} - \frac{\sqrt{2}}{2} \right) x = 1  \implies  x = \frac{1}{\frac{1}{\sqrt{2}} - \frac{\sqrt{2}}{2}}.
\end{equation}
For any canonical representation of \(\frac{1}{\sqrt{2}} - \frac{\sqrt{2}}{2} \) would simplify to zero and the simplified version of the equation
is \( 0 x  = 1\) has an empty solution set.

A related, but not identical notion of simplification requires that every expression that is algebraically equivalent to zero must simplify to zero. Such a representation is called \emph{normal}.  A normal representation of $ x+y+1$ could be any one of $1+x+y$ or $x + 1 + y$. So a normal representation needn't be a canonical represenation. 

\subsection*{Canonical simplification}



\paragraph{Rational numbers} One the set of rational numbers, a (not the) possible canonical representation  is to write every rational number as either an integer or as an improper
rational number.  Thus in this scheme, the canonical representation of $\frac{1812}{32}$ is  $\frac{453}{8}$. The process reducing a rational
number to its reduced rational  form is algorithmic--there is no guessing or luck required, only a finite number of steps that  involve only arithmetic are needed.

The reduced rational form isn't the only canonical form, there are others. For example, until the 21st century, stock prices in the US were listed as mixed fractions; for example, $10 \frac{2}{3}$ dollars per share. And food recipes still use this scheme as well ($2 \frac{1}{3}$ cups of butter). This scheme 
breaks the  convention that in algebra juxtaposition means multiplication, so its use in algebra problematic, so its use outside of cookbooks and stock tables is rare. But
nevertheless the representation is canonical.

All this is not as tidy as described. We've claimed that the reduced rational form is a canonical. That much is true. We've also claimed that the 
process for finding the canonical form is algorithmic. Without additional conditions, that's false.  It's a famous fact that 
$\sum_{k=1}^\infty \frac{1}{k^2}  = \frac{\uppi}{6}$. So we have
\begin{equation}
\frac{1}{\uppi^2} \sum_{k=1}^\infty \frac{1}{k^2}  \in \rationals
\end{equation}
Although the canonical representation of $\frac{1}{\uppi^2} \sum_{k=1}^\infty \frac{1}{k^2}  $ is 6, the process of discovering this isn't a matter of 
using an algorithm to find a greatest common divisor.  So as a practical matter, although the reduced rational form is a canonical representation for the
set of rational numbers, we don't have an algorithm that can find it.  To distinguish such things, we say that a number such as $\frac{1812}{32}$  is
an explicit representation of a rational number, but $\frac{1}{\uppi^2} \sum_{k=1}^\infty \frac{1}{k^2}  $ is an \emph{implicit} representation of
a rational number.

\paragraph{Algebraic  numbers} A number is \emph{algebraic} if it is root of a nonzero polynomial with integer coefficients. Every rational number, for example 
$p/q$, where $p$ and $q$ are integers and $q$ is nonzero is an algebraic number; since $1/\sqrt{2}$ is a zero of the polynomial $2 x^2 - 1$, it
follows that $1/\sqrt{2}$ is an algebraic number. 

Some, but not all algebraic numbers, have a \emph{closed form} that only involves finite sums, products, and quotients along with integer roots. For example, the roots of 
$x^2 - 10 x + 23$, specifically the numbers $5-\sqrt{2}$ and $\sqrt{2}+5$, are both algebraic numbers that are in closed form.  As far as I know, there is no known
algorithm 

 The number $\frac{1}{\sqrt{2} + \sqrt{3}$ is in closed form. To show that it's an algebraic number, we need
to find a polynomial with integer coefficn
\begin{equation}
  5 + \sqrt{2}, \quad \sqrt{5 + \sqrt{2}}, \quad \frac{1}{\sqrt{2} + \sqrt{3} + \sqrt{5}}.
\end{equation}


\paragraph{Polynomials} Moving from rational numbers to to polynomials in one variable with coefficients that are rational numbers,  a possible canonical representation is to 
require that
\begin{alphalist}
  \item The polynomials is fully expanded.
  \item The terms of the polynomial are be ordered from high to low power with each power of the variable appearing at most one time.
  \item All coefficients must be expressed in reduced rational form
  \item In each term, the coefficient is before the variable term.
\end{alphalist}
Algorithmically, given any polynomial with constant terms that are explicitly rational numbers, with sufficient patience we can find its canonical representation. Here we show some polynomials and their canonical representations: 

\begin{figure}[h]
\center
\begin{tabular}{|m{5.18cm}|m{7.18cm}|@{}m{0pt}@{}} \hline 
\textbf{Polynomial}  & \textbf{Canonical form} \\[7pt]  \hline
$  1 +  x \frac{1812}{32}$   &  $  \frac{453}{8} x + 1 $ \\[7pt]  \hline 
$ 15 - 8 x$  & $-8 x + 15$ \\[7pt]  \hline 
$\left( x-5\right) \, \left( x-3\right) -{{x}^{2}} $  & $-8 x + 15$  \\[7pt]  \hline 
$   (x+1)(x+2) $ &  $ x^2 + 3 x + 2$  \\[7pt]  \hline 
$   {{\left( x-1\right) }^{6}}-1  $  & ${{x}^{6}}-6 {{x}^{5}}+15 {{x}^{4}}-20 {{x}^{3}}+15 {{x}^{2}}-6 x $ \\[7pt]  \hline
\end{tabular}
\caption{Polynomials and their canonical representations}
\end{figure}
Although ${{x}^{6}}-6 {{x}^{5}}+15 {{x}^{4}}-20 {{x}^{3}}+15 {{x}^{2}}-6 x $ is the canonical representation of ${{\left( x-1\right) }^{6}}-1$, it's
hard to argue that for this case that the canonical representation is \emph{simpler} than ${{\left( x-1\right) }^{6}}-1$. For example, if your ultimate goal is to  numerically evaluate the polynomial  ${{\left( x-1\right) }^{6}}-1$, the entering the partial factored expression into a calculator is far easier
than using the canonical representation.  Similarly, if the goal is to solve an equation, solving
\begin{equation*}
  {{\left( x-1\right) }^{6}}-1 = 0
\end{equation*}
is a far easier problem than is solving
\begin{equation*}
  {{x}^{6}}-6 {{x}^{5}}+15 {{x}^{4}}-20 {{x}^{3}}+15 {{x}^{2}}-6 x  = 0.
\end{equation*}
Even if you got lucky and were able to factor the left side, namely
\begin{equation*}
\left( x-2\right)  x\, \left( {{x}^{2}}-3 x+3\right) \, \left( {{x}^{2}}-x+1\right)  = 0
\end{equation*}
you are still stuck solving two quadractic equations. 


As nice as it might be to make all semantically equal polynomials syntatically equal, it's not always what we really want to do.  Nobody
really wants to see all 101 terms of the fully expanded version of $(2  x-1)^100$.

\begin{comment}
You needn't be especially cynical to speculate that the only
reason for the requirement of simplified answers is to reduce the 
teacher's burden of grading papers--it is much easier to simply mark 
any answer that isn't visually identical to the answer key as wrong.
\end{comment}
Compounding the frustration is that some simplification rules, such as eliminating radicals in the 
denominator might seem arbitrary, turning the process of simplification 
into a silly game. Why is $\sqrt{2}/2$ a simplification of 
of $1/\sqrt{2}$? Certainly if you need a decimal approximation, entering
$1/\sqrt{2}$ into a calculator takes no more effort than does $\sqrt{2}/2$. Although
that's true today, it wasn't true before the era of electronic calculators. Long
ago doing the long division problem 
\[
   1.414213562373095\cdots \, \, \showdiv{\, \, 1.000000000000000000}
\]
by hand was odious, but doing the algebraically identical calculation 
(\(\frac{\sqrt{2}}{2} \))
\[
   2.0000000000000\cdots \,\, \showdiv{\, \, 1.414213562373095\cdots}
 \]
 has always been a snap. That fact tipped the scale, so it's standard that promoting
 radicals from the denominator to the numerator results in a simplified
 expression.

 Promoting a single factor of a radical in a denominator to the numerator
 is arguably a nice skill to learn, but what about a sum of two or more
 radicals? Possibly you were taught the process for a sum of two radicals; for
 example,
 \begin{equation}
   \frac{1}{\sqrt{3}+\sqrt{2}} = \sqrt{3}-\sqrt{2}.
 \end{equation}
But what about a sum of three radicals? What is the simplification of
\begin{equation}
   \frac{1}{\sqrt{7}+\sqrt{5}+\sqrt{3}+\sqrt{2}} \mbox{?}
\end{equation}
\begin{equation}
   \frac{22 \sqrt{105}- 34 \sqrt{70}- 50 \sqrt{42}+62 \sqrt{30}+135 \sqrt{7}- 133 \sqrt{5}-145 \sqrt{3}+185 \sqrt{2}}{215}.
\end{equation}


 Even for the question if a polynomial should simplified by expanding
 or factoring is unclear. If the goal of simplification is reduce the
 number of terms in an expression, it seems that we should favor 
 factoring over expanding. Indeed, factoring ${{x}^{3}}-8 {{x}^{2}}+21 x-18$
 yields ${{\left( x-3\right) }^{2}} \left( x-2\right)$, which has fewer
 terms and is arguably simpler. 

But if the goal of of simplification is to make all algebraically
identical expressions (also known as semanatically equal) to be syntactically the same, it's a fool's 
errand. To illustrate, although
\begin{equation}
    (a-b) \sin(a-b) = (b-a) \sin(b-a)
\end{equation}
is an identity, it's hard to argue that one expression is more simple than
the other. Sure, we could so devise an rule (similar to alphabetizing) that
would determine that \((a-b) \sin(a-b)\) is more simple than \((b-a) \sin(b-a)\),
but it hardly seems worth the effort. Similarly, we would need to decide
if \(1+x+x^2\) is more simple than \(x^2 + x + 1\). Again, rules for
such things hardly seems worth the effort.

Actually, its more than a fool's errand. Richardson's theorem tells us
that once we allow expressions to involve trigonometric and exponential 
functions, the absolute value function, rational numbers, and constants form by apply the logarithm
to a number, there is no algorithm that can prove always

\url{https://en.wikipedia.org/wiki/Richardson%27s_theorem}


What it means to be 
simplified can be context dependent. But there are some simplifications that nearly 
everybody agrees should generally be done. These are:

\begin{tcolorbox}
\begin{alphalist}

\item  Reduce all rational numbers to lowest terms.

\item All arithmetic in sums, products, and exponents should be done.

\item All common additive and multiplicative terms should be combined.

\item For any real valued expression, use the identities  $1 \times x   = x ,  0  x = 0, 1^x = 1$ and  $x^1 =x$ to replace the left side by the right side.

\item Provided $x$ is a nonzero and real valued expression,  use the identities $\frac{x}{x} = 1, x^0 = 1$ to replace the left side by the right side.

\item Provided $x$ is a nonnegative and real valued expression, use the identity $\left(x^a\right)^b = x^{a b}$  to replace the left side by the right side.

\item Use the well known values of the trigonometric functions at the integer multiplies of $\uppi/6$ and $\uppi/4$ to simplify these values.

\item For any odd function $O$, replace $O(x) + O(-x)$ by zero. For any odd function $E$, replace $E(x) - E(-x)$ by zero.

\item Use the well known values of the logarithms to simplify these values.

\item For a positive integer $n$, replace $\frac{1}{\sqrt{n}}$ by $\frac{\sqrt{n}}{n}$.

\item For a positive integers $m$ and $n$, replace $\sqrt{m n^2}$ by $n \sqrt{m}$. 

\end{alphalist}
\end{tcolorbox}

\noindent In all of these rules, $x$ can match any expression or any subexpression, not just an explicit match to the variable $x$.  And sometimes, we may need to use factoring some other identities to find a match.  Thus our guideline is a guideline, not an algorithm.  




\begin{itemize}
\item The subexpressions of the quotient  $\displaystyle \frac{x (x^2+1)}{x^2 + 1}$ are $x (x^2+1)$ and $x^2 + 1$, but
$\frac{x^2+1}{x^2 + 1}$ isn't a subexpression.  But rearranging the quotient $\displaystyle \frac{x (x^2+1)}{x^2 + 1}$ to the product $\displaystyle x \frac{x^2+1}{x^2 + 1}$, we now have $ \frac{x^2+1}{x^2 + 1}$ as a subexpression of the product, and this subexpression 
explicitly matches rule `e.' So we simplify $\displaystyle \frac{x (x^2+1)}{x^2 + 1}$ to $x$.

\item Although there are no common additive terms in $ 6 |x| - |28 x|$, there are if we use the identity $|x y| = |x| |y|$  Applying this identity
first yields
\[
    6 |x| - |28 x| = 6 |x| - |28| |x| = 6 |x| - 28 |x| = -22 |x|.
\]

\item The expression $\sqrt{50}$ doesn't match any of these rules.  But it does match the last rule if we use the factorization $50 = 2 \times 5^2$.

\end{itemize}



\noindent 

\paragraph{Examples} 

\begin{alphalist}

\item Using rules 	`a' through `d,' we would simplify
\[
   \frac{2}{3} +  6^3 + (x+1)^1 + 0 \times x^2  + 1^{10^9}  + z - 107 z = \frac{656}{3} + x - 106 z.
\]
The ordering of the terms in the sum $\frac{656}{3} + x - 106 z$ is a detail that few teachers would require.

\item Using rule `e,' we would simplify
\[
    \frac{46 (x^2+1)}{x^2 + 1} + (|x| + 1)^0 = 46 + 1 = 47.
\]
Here both $x^2 + 1$ and $|x| + 1$ are nonzero and real valued.   When it's not certain that $x$ matches with a nonzero expression and we use
these simplications, we should make a note of the assumptions; for example:
\[
     28 \frac{x + 1}{x+ 1}  = 28, \mbox{ provided } x+ 1 \neq 0.
\]
And
\(
     0^{x - 216} = 1,  \mbox{ provided } x - 216 \neq 0.
\)
Actually, the question of whether or not $0^0$ is undefined or if it is equal to $1$ is controversial. 

\item Using rule `f,' we would simplify
\[
    \sqrt{x^2+1}^2 =  \left( (x^2 + 1)^{1/2} \right)^2 = x^2 + 1.
\]
Again, if it's not certain that $x$ matches with a nonnegative real valued expression, our work should note the assumption; for example:
\(    \sqrt{x}^2 = x, \mbox{ provided } x  \geq 0\).  

\item Using rules `g' and `h,' we have
\[
   \cos(5 \uppi/3) + \log_{10}(100) = \frac{5}{2}.
\]

\item Using rule `i,' we have
\[
    \frac{107}{\sqrt{5}} = \frac{107 \sqrt{5}}{5}.
\]
\end{alphalist}

Please be careful with rule `f.'  Using this rule without adhering to the condition yields rubbish:
\[
   \left( (-1)^2 \right)^{1/2} = 1^{1/2} = 1.
\]
But 
\[
   (-1)^{2 \times \frac{1}{2}} = (-1)^1 = -1.
\]
So the proviso isn't optional.  

\paragraph{Non Examples} 

\begin{alphalist}

\item Our simplification guidelines say nothing about expanding or factoring a polynomial. So by our standard, both $(x-1)(x+1)$ and $x^2 - 1$ are simplified.

\item Our simplification guide also says that both $|x y|$ and $|x| |y|$ are simplified. But maybe we should append a rule for this case.

\item Although \(\sin(-x) = - \sin(x)\), our simplification guide says that both $\sin(x)$ and $\sin(-x)$ are simplified.  Similarly both
$\sin(x-x^2)$ and $-\sin(x^2-x)$ are simplified, but algebraically equivalent. But we do have a rule that requires
that we simplify $\sin(x) + \sin(-x)$ to zero.




\item 
\end{alphalist}








Likely you have been taught that to simplify a quotient with a radical in the demonontator such as $\frac{1}{\sqrt{2}}$, you need to multiply by a well chosen representation of one; for example
\[
   \frac{1}{\sqrt{2}} = \frac{1}{\sqrt{2}} \times \frac{\sqrt{2}}{\sqrt{2}}  = \frac{\sqrt{2}}{2}.
\]
And you were told that \(\frac{\sqrt{2}}{2}\) is a simplification of \(\frac{1}{\sqrt{2}} \). 
Long ago, before when calculators were exotic, doing the long division of \(1 \div 1.414213562373095\dots\) was tedious, but
the equivalent calcuatoin of \(1.414213562373095 \dots \div 2\) is easy.
\[
   1.414213562373095 \showdiv{1.000000000000000000}
\]
but the long division
\[
     2 \showdiv{1.414213562373095 \dots} 
\]
is easy.
\[
56 \showdiv{3678}
\]

\[
\frac{1}{\sqrt{3}+\sqrt{5} + \sqrt{7}} = \frac{\sqrt{7}+ 5 \sqrt{3} +\sqrt{243} - 2 \sqrt{105}}{59}
\]
\subsection*{Ideally}

From a teacher's perspective,  we'd like to make paper grading as fast as possible. And one way to sped that task is for all correct answer to look exactly alike, and for all wrong answers to not look exactly like the correct answer.  If we stick to answers that are polynomials with coefficients that are explicit 
rational numbers, we can achieve this by requiring that all answers be fully expanded with coefficients expressed as improver rational numbers in reduced form and arranged from low to high power.  And example of a simplified expression would be
\[
     3 + 5 x - \frac{107}{46} x^2.
\] 
Under these rules this answer is correct and no other answer, including the algebraically equivalent \( - \frac{107}{46} x^2 + 5 x + 3\) is wrong. Imposing this

\subsection*{Efficiency}

(Horner's method)

\subsection*{Accuracy} 


And if  you have studied a bit more physics and learned about Einstien's  special theory of relativity, you might remember that the kenetic energy is given by 
\[
    T  =  m c^2 \left (\frac{1}{\sqrt{1- v^2/c^2}} - 1\right),
\]
where $c$ is the speed of light.  Let's find $T$ given $m = 1, c =  299792458$ and $v= 10^{-8} c$.  Pasting in these values into a calculator that uses about 15 decimal digits, we get
\[
   T =  299792458^2 \left(\frac{1}{1 - \sqrt{1-10^{-16}}} -1 \right) = 299792458^2 \times 2.220446049250313 {{10}^{-16}} = 19.95637385869426.
\]



\end{document}

%If we were to append a rule for this (and many similar cases of an odd function), we would need to decide if $\sin(x-x^2)$ should be simplified to $-%\sin(x^2-x)$, for example.  We won't do that.\footnote{The Maxima computer algebra system has a rule for simplifying odd functions. To do this, compares the expressions $x - x^2$ and $x^2-x$ in a way that is similar to alphabetizing.} Similarly, for an even function such as $\cos$, we could have a rule that decides if $\cos(x-x^2)$ simplifies to $\cos(x^2-x)$, for example.