\documentclass[12pt]{article}
\usepackage{wasysym}
\usepackage{phonenumbers}
\usepackage{xcolor}
\usepackage[UKenglish]{isodate}% http://ctan.org/pkg/isodate
\usepackage[paperwidth=8.5in,paperheight=11in,margin=0.75in]{geometry} 
\usepackage[USenglish]{babel}
\usepackage{hyperref}
\usepackage[activate={true,nocompatibility},final,tracking=true,kerning=true,spacing=true,factor=1100,stretch=10,shrink=10]{microtype}
\frenchspacing
\usepackage[nodayofweek,level]{datetime}
\usepackage{calc,url}
\newcounter{qz}\setcounter{qz}{0}
\newcommand{\qz}{%\
\setcounter{qz}{\value{qz}+1}
\textbf{In-class  \theqz} \,}

\newcounter{hw}\setcounter{hw}{0}
\newcommand{\hw}{%\
\setcounter{hw}{\value{hw}+1}
\textbf{HW \thehw} \,\,}

\newcounter{ex}\setcounter{ex}{0}
\newcommand{\ex}{%\
\setcounter{ex}{\value{ex}+1}
Exam \theex}

\usepackage[T1]{fontenc} 
\usepackage{fourier}
%\usepackage{tgschola} %to look retro
\newenvironment{mypar}[2]
  {\begin{list}{}%
    {\setlength\leftmargin{#1}
    \setlength\rightmargin{#2}}
    \item[]}
  {\end{list}}


\newcounter{wk}\setcounter{wk}{0}
\newcommand{\wk}{%\
\setcounter{wk}{\value{wk}+1}
\thewk \,\,}

\newcommand{\term}{Fall }

\usepackage{enumerate}
\usepackage{graphicx}

\usepackage{paralist}
\renewenvironment{description}[0]{\begin{compactdesc}}{\end{compactdesc}}

\newenvironment{alphalist}{
  \begin{enumerate}[(a)]
    \addtolength{\itemsep}{-0.5\itemsep}}
  {\end{enumerate}}
  \cleanlookdateon% Remove ordinal day reference
  \newcommand{\RomanNumeralCaps}[1]
      {\MakeUppercase{\romannumeral #1}}

\newcommand{\coursename}{Plane Trigonometry}
\newcommand{\coursenumber}{MATH 103}
\newcommand{\sectionnumber}{01}

\begin{document}
\large
\begin{center}
    \textbf{\coursename}  \\
    {\coursenumber--\sectionnumber} \\
     {\term \the\year} \\
\end{center}

\vskip0.25in
\normalsize


\begin{center}
\begin{description}
    \item[Instructor:] Dr.\  Willis, Professor of Mathematics
    \item[Office:]  Discovery Hall, Room 368
    \item[\phone:]  \phonenumber[country=US]{3088658868}
    \item[Email:] \href{mailto:willisb@unk.edu}{willisb@unk.edu}
    \item[Zoom for classes:] For Zoom class meetings, use the Meeting ID: 616 568 5706. 
    \item[Zoom for office hours:] For Zoom office hours, use the Meeting ID: 981 7908 2161 
    \item[Office Hours:] Either in person or by Zoom: Monday, Wednesday, and  Friday, \mbox{9:30--11:00}; Tuesday and Thursday 13:00 -- 14:00; Monday and Wednesday 13:30 -- 15:00;  and by appointment.
  \end{description}
\end{center}

\subsubsection*{Textbook \& WebAssign}

The Class Key for WebAssign is \textbf{unk 6035 9620}. Our textbook is
\emph{Trigonometry (LL)-W/WebAssign Access}, by Larson (ISBN: 9781337605168).
You \emph{must} register for WebAssign.

\subsubsection*{Important Dates}

\begin{mypar}{0.25in}{0.25in}
    \textbf{First online homework} \dotfill Friday, 3 September \\
    \textbf{Exam 1} \dotfill Tuesday, 21 September \\
    \textbf{Exam 2} \dotfill Tuesday, 26 October \\
    \textbf{Exam 3} \dotfill Tuesday, 30 November \\
     \textbf{Final exam} \dotfill Thursday 16 December, 8:00--10:00
\end{mypar}

\subsubsection*{Grading}

Your course grade will be based on online homework, in class work, midterm exams, and a comprehensive 
final exam; specifically:
\begin{mypar}{0.25in}{0.25in}
    \textbf{Online homework:} \emph{29 five point assignments} \dotfill 145 (total)\\
    \textbf{In class work:}  \emph{12 ten point assignments}  \dotfill 120 (total) \\
    \textbf{Mid-term exams 1,2, and 3:} \emph{100 points each} \dotfill 300 (total)\\
      \textbf{Comprehensive Final exam} \dotfill 150 (total)\\
\end{mypar}
\noindent The following table shows the \emph{minimum} number of points (out of 715) that
are required for each of the twelve letter grades D- through A+. For
example, a point total of 620 points will earn you a grade of B+ and 
a point total of 619 points will earn you a grade of B. A point
total of 429 or less earns you a failing course grade.
%[[429,"D-"],[453,"D"],[477,"D+"],[501,"C-"],[525,"C"],[549,"C+"],[572,"B-"],[596,"B"],[620,"B+"],[644,"A-"],[668,"A"],[692,"A+"]]
\begin{mypar}{0.25in}{0.25in}
    \begin{minipage}{2.5in}
        D-  \dotfill 429  \\
        D \dotfill 453 \\
        D+ \dotfill 477 \\
        C- \dotfill 501 \\
        C \dotfill 525 \\
        C+ \dotfill 549 
    \end{minipage}
    \phantom{xxx}
    \begin{minipage}{2.5in}
        B- \dotfill 572 \\
        B \dotfill 596 \\
        B+ \dotfill 620 \\
        A- \dotfill 668  \\
        A \dotfill  692 \\
        A+ \dotfill  750 
    \end{minipage}
\end{mypar} 

\subsubsection*{Course Calendar}

Generally, we'll adhere to the scheduled exam dates even if we are ahead or behind with course work.  When we are ahead or behind, the
topics on the exams will be appropriately adjusted.  There is no new topics scheduled for dead week, if we adhere to the schedule, we'll review during dead week, but
if we fall behind, we'll cover new topics during dead week.


\vspace{0.1in}
\noindent \textbf{Notices:}


\begin{alphalist}
   \item Exams will be given on the Tuesday of the week they are assigned.
   
   \item In class work  will generally be 
    done on Tuesday of the week they are assigned.

    \item Online homework (labeled \textbf{HW}) will be due at midnight on
          Friday of the week they are assigned. Most homework assignments will
          cover multiple sections. 

    \item The homework assignment that is due the Friday after Thanksgiving
          will be assigned sufficiently early for you to complete it before
          the Thanksgiving break.
\end{alphalist}

\vspace{0.1in}

\begin{center}
    \small
\begin{tabular}  {|l|l|l|l|l|}
\hline
{\bf Week}  & \textbf{Monday} &  {\bf Section(s)} & {\bf Topic(s)} & \textbf{Assessments} \\
\hline \hline 
\wk    & 8/23 &    \S P.1 -- \S P.10   & Prerequisites & \qz  \\
\wk    & 8/30  &  \S1.1 -- \S1.2  &  Radian measure, Trigonometric functions: unit circle  & \qz \hw  \\
\wk    & 9/6 &     \S1.3 --\S1.4  &   Right triangles, Trig functions  & \qz \hw \\
\wk    & 9/13  &     \S1.5 -- \S1.6  & Graphs of sine \& cosine; Graphs of other trig  & \qz \hw            \\
\wk    & 9/20 &  \S1.7--\S1.8    &  Inverse trig functions, applications    & \textbf{\ex} \hw \\ \hline
\wk    & 9/27   & \S2.1 -- \S2.2   & Fundamental identities,  Verifying identities &  \qz \hw  \\
\wk    & 10/4     & \S2.3 -- \S2.4  &  Solving trig equations, sum \& difference formulas & \qz \hw  \\
\wk   & 10/11   & \S2.5 -- \S3.1  &  Multiple angle formulas \& Law of sines     & \qz \hw  \\
\wk  &  10/18   & \S3.2 -- \S3.3 &  Law of cosines,  Vectors   & \qz  \hw \\ 
\wk &  10/25     &   \S3.4& Dot products  &  \textbf{\ex}  \hw \\ \hline
\wk  & 11/1  &   \S4.1 -- \S4.2 & Complex numbers \&  Complex solution of equations   & \qz  \hw  \\
\wk   & 11/8  & \S4.3 --\S4.4  & Complex plane \&  Trig form of numbers    \&  & \qz \hw \\
\wk   & 11/15& \S4.5  &  DeMoivre's theorem    & \qz \hw  \\
\wk   & 11/22   &  \S6.5, \S6.7     & Rotations \& Polar coordinates & \qz  \hw   \\
\wk   & 11/29    &  \S6.8      &  Graphs of polar equations   & \textbf{\ex} \hw   \\ \hline
\wk   & 12/6      &  & Catch up or Review        & (none) \\  \hline
\wk   & 12/13      &  &     \hfill  & \textbf{ Final Exam}  \\  \hline
   
\end{tabular}
\end{center}

\newpage

\subsubsection*{Additional Resources}

\begin{alphalist}


\item Reliable Internet access.

\item An Internet connected camera (for turning in class work electronically).

\item  An Internet connected computer (not just a phone or tablet) that can run Zoom. 

\item If we need to convert this class to remote learning, your computer will need to have a microphone and a camera. For remote office hours, it can be useful to have a separate camera that can be pointed toward a well-lit writing surface.

\item A basic scientific calculator (needn't be a graphing calculator).

\item Pencils, erasers, notebook for note taking. Colored pens or pencils are nice for note taking.

\end{alphalist}



\subsubsection*{In-class work \& online homework}

Except for examination days, we will do in class work for a portion 
of each class on Tuesday. In class work must be turned in 
electronically to Canvas (not emailed to me) by midnight the day we 
do it. Online homework is due each Friday at midnight. 

\subsubsection*{Online classes}

If you are ill, please let me know and join class via Zoom. But 
be aware that technology doesn't always work, sometimes I forget to 
click all the buttons to make it work, and the readability of class 
materials over Zoom is sometimes poor. So if you join class regularly 
by Zoom, it's your choice, but I do not recommend it. 

\subsubsection* {Policies}

\begin{enumerate}

\item Class cancellations due to weather or illness or other 
unplanned circumstances may require that we make minor adjustments
to the course calendar, exam dates, and due dates or specifics for 
course assessments. Should we end the term with a point total that
differs from 715, your point total will be scaled to 715 points and
we will use the course grade scheme in the section `Grading.'

\item You must register for WebAssign in time to complete the first 
assignment (due midnight 3 September).

\item Extra credit is not allowed. 

\item For online homework, you may work in groups and you may 
seek help from the Learning Commons. 

\item For examinations, you make use a teacher provided crib sheet, 
but no other reference materials. You may also use a pencil, eraser, 
and a scientific calculator. For examinations, your phone and all such
devices must be turned off and \emph{out of sight}. Checking your phone
to look at the time is \emph{not} allowed. Using unauthorized 
materials during an examination will earn you a failing course grade.

\item Generally, if you are ill or absent for any reason (including 
athletics), you must turn in your in class work on time. Permission to
turn in work late must be made in advance, otherwise late in class work 
will count zero points.

\item Generally,  if you are ill or absent for any reason (including 
athletics), you must turn in your online homework on time. Permission to
turn in work online homework late must be made in advance, otherwise it will
count zero points.
 

\item During class time, please refrain from using with electronic devices. If your 
device usage distracts your classmates, I will ask you to put it away. If it's my 
impression that you are often not paying attention in class, I reserve the right to 
decline to help you during office hours.

\item The final examination will be \emph{comprehensive} and it will be given during the 
time scheduled by the University. Except for \emph{extraordinary circumstances}
you must take the exam at this time.


 
\item If you have questions about how your work has been graded, make an appointment with me immediately.

\item All printed materials, in either paper or digital form, that I 
provide for you in this class, are for your own use. Re-posting or 
sharing these materials with other persons is prohibited. 

\item Please regularly check Canvas  to verify that your scores have 
been recorded correctly.  If I made a mistake in recording one of
your grades, I'll correct it provided you saved your paper.

\item The work you turn in is expected to be \emph{accurate, 
complete, concise, neat}, and \emph{well-organized}.  
\emph{You will not earn full credit on work that falls short of 
these expectations.}

\item For examinations, show your work.  No credit will be given for multi-step problems without the necessary work. Your solution must contain enough detail
so that I am convinced that you could correctly work any similar problem. Also erase or clearly mark any work you want me to ignore; otherwise,
I'll grade it.  

\end{enumerate}

\paragraph{Course Prerequisites} MATH 102 or Math ACT Score of 22 or greater and two years of high school algebra.

\subsubsection*{Course Description}  Plane Trigonometry.  3 credit hours. This course is the study of trigonometric functions.

\subsubsection*{General Studies Program Information}

MATH 103 is a general studies course that satisfies LOPER 4 (Mathematics, Statistics and Quantitative Reasoning) foundational requirement. 

\subsubsection*{Purpose of General Studies} The UNK LOPERs General Studies Program helps students to develop core academic skills in collecting and using information, communications in speech and writing, and quantitative reasoning (LOPERs 1-4); to acquire broad knowledge in a variety of disciplines across the arts, humanities, social sciences, and natural sciences (LOPERs 5-8); and to instill dispositions that prepare students to lead responsible and productive lives in a democratic, multicultural society (LOPERs 9-11).

\subsubsection*{GS Foundational Requirement Program Objective} Courses 
are designed for students to develop core academic skills in 
collecting and using information, communications in speech and 
writing, and quantitative reasoning.

LOPER 4 (Mathematics, Statistics, and Quantitative Reasoning) Learning Outcomes: 
\begin{alphalist}
    \item  Can describe problems using mathematical, statistical, or programming language 
     \item Can solve problems using mathematical, statistical, or programming techniques 
    \item Can construct logical arguments using mathematical, statistical, or programming concepts 
     \item Can interpret and express numerical data or graphical information using mathematical, statistical, or programming concepts and methods
        \end{alphalist}

The first Learning outcome is met by the mathematical set up and 
preparation of solutions to various problems encountered in this 
course about trigonometric functions and trigonometric identities 
as well as their applications. The second and third learning 
outcomes are met in solving such problems using mathematical 
skills and logical arguments. The fourth learning outcome is met 
by understanding those problems which involve graphs and data and by 
giving solutions to those problems. The four learning outcomes are 
assessed by grading homework, quizzes, exams, and/or projects based 
on the set up and defense of the submitted work, the validity of the 
submitted solution’s logical reasoning, the accuracy of the answers, 
the accuracy of the graphs and data in the submitted solutions 
and/or the accuracy of the interpretation 
of the graphs and data from the assigned problem.

\subsubsection*{Academic integrity} Students are expected to adhere to 
the UNK Academic Integrity Policy found in the current Undergraduate 
Academic Catalog: 
\url{https://catalog.unk.edu/undergraduate/academics/academic-regulations/academic-integrity-policy/}

\subsubsection*{Students with Disabilities or Those Who are Pregnant}

\paragraph{Students with Disabilities} It is the policy of the University of Nebraska 
at Kearney to provide flexible and individualized reasonable 
accommodation to students with documented disabilities. To receive 
accommodation services for a disability, students must be
registered with the UNK Disabilities Services for Students (DSS) 
office, 175 Memorial Student Affairs Building, 
\phonenumber[country=US]{3088658214} or 
by email \href{mailto:unkdso@unk.edu}{unkdso@unk.edu}

\paragraph{UNK Statement of Diversity \& Inclusion:} UNK stands in solidarity and unity with our students of color, our Latinx and international students, our LGBTQIA+ students and students from other marginalized groups in opposition to racism and prejudice in any form, wherever it may exist. It is the job of institutions of higher education, indeed their duty, to provide a haven for the safe and meaningful exchange of ideas and to support peaceful disagreement and discussion. In our classes, we strive to maintain a positive learning environment based upon open communication and mutual respect. UNK does not discriminate on the basis of race, color, national origin, age, religion, sex, gender, sexual orientation, disability or political affiliation. Respect for the diversity of our backgrounds and varied 
life experiences is essential to learning from our similarities as well as our differences. The following link provides resources and other information regarding D\&I: 
\url{https://www.unk.edu/about/equity-access-diversity.php}

\paragraph{Students Who are Pregnant} It is the policy of the University of Nebraska at Kearney to provide flexible and individualized reasonable accommodation to students who are pregnant. To receive accommodation services due to pregnancy, students must contact Cindy Ference in Student Health, 308-865-8219. The following link provides information for students and faculty regarding pregnancy rights.\footnote{\tiny  \url{http://www.nwlc.org/resource/pregnant-and-parenting-students-rights-faqs-college-and-graduate-students}}

\paragraph{Reporting Student Sexual Harassment, Sexual Violence or Sexual Assault} Reporting allegations of rape, domestic violence, dating violence, sexual assault, sexual harassment, and stalking enables the University to promptly provide support to the impacted student(s), and to take appropriate action to prevent a recurrence of such sexual misconduct and protect the campus community. Confidentiality will be respected to the greatest degree possible. Any student who believes she or he may be the victim of sexual misconduct is encouraged to report to one or more of the following resources:
\begin{alphalist}
\item Local Domestic Violence, Sexual Assault Advocacy Agency \phonenumber[country=US]{3082372599}

\item Campus Police (or Security) \phonenumber[country=US]{3088658911}

\item Title \RomanNumeralCaps{9} Coordinator \phonenumber[country=US]{3088658655}

\end{alphalist}
Retaliation against the student making the report, whether by students or University employees, will not be tolerated.If you have questions regarding the information in this email please 
contact Mary Chinnock Petroski, Chief Compliance Officer (
   \href{mailto:petroskimj@unk.edu}{petroskimj@unk.edu} 
    or phone \phonenumber[country=US]{3088658400}.


\end{document}

